\documentclass[12pt,letterpaper]{article}
\usepackage[utf8]{inputenc}
\usepackage[version=3]{mhchem}
\usepackage[journal=jacs]{chemstyle}
\usepackage{amsmath}
\usepackage{amsfonts}
\usepackage{amssymb}
\usepackage{makeidx}
\usepackage{xcolor}
\usepackage[stable]{footmisc}
\usepackage[section]{placeins}
\usepackage{longtable}
\usepackage{array}
\usepackage{xtab}
\usepackage{multirow}
\usepackage{colortab}
\usepackage{siunitx}
\sisetup{mode=text, output-decimal-marker = {,}, per-mode = symbol, qualifier-mode = phrase, qualifier-phrase = { de }, list-units = brackets, range-units = brackets, range-phrase = --}
\DeclareSIUnit[number-unit-product = \;] \atmosphere{atm}
\DeclareSIUnit[number-unit-product = \;] \pound{lb}
\DeclareSIUnit[number-unit-product = \;] \inch{"}
\DeclareSIUnit[number-unit-product = \;] \foot{ft}
\DeclareSIUnit[number-unit-product = \;] \yard{yd}
\DeclareSIUnit[number-unit-product = \;] \mile{mi}
\DeclareSIUnit[number-unit-product = \;] \pint{pt}
\DeclareSIUnit[number-unit-product = \;] \quart{qt}
\DeclareSIUnit[number-unit-product = \;] \flounce{fl-oz}
\DeclareSIUnit[number-unit-product = \;] \ounce{oz}
\DeclareSIUnit[number-unit-product = \;] \degreeFahrenheit{\SIUnitSymbolDegree F}
\DeclareSIUnit[number-unit-product = \;] \degreeRankine{\SIUnitSymbolDegree R}
\DeclareSIUnit[number-unit-product = \;] \usgallon{galón}
\DeclareSIUnit[number-unit-product = \;] \uma{uma}
\DeclareSIUnit[number-unit-product = \;] \ppm{ppm}
\DeclareSIUnit[number-unit-product = \;] \eqg{eq-g}
\DeclareSIUnit[number-unit-product = \;] \normal{\eqg\per\liter\of{solución}}
\DeclareSIUnit[number-unit-product = \;] \molal{\mole\per\kilo\gram\of{solvente}}
\usepackage{cancel}
\usepackage{graphicx}
\usepackage{lmodern}
\usepackage{fancyhdr}
\usepackage[left=4cm,right=2cm,top=3cm,bottom=3cm]{geometry}

\usepackage[backend=bibtex,style=chem-acs,biblabel=dot]{biblatex}
\addbibresource{references.bib}

\usepackage{titlesec}
\usepackage{enumitem}
\titleformat*{\section}{\bfseries\large}
\titleformat*{\subsection}{\bfseries\normalsize}

\usepackage{float}
\floatstyle{plaintop}
\newfloat{anexo}{thp}{anx}
\floatname{anexo}{Anexo}
\restylefloat{anexo}
\restylefloat{figure}

\usepackage[margin=10pt,labelfont=bf]{caption}

\usepackage{todonotes}

\usepackage[colorlinks=true, 
            linkcolor = blue,
            urlcolor  = blue,
            citecolor = black,
            anchorcolor = blue]{hyperref}


\begin{document}
\renewcommand{\labelitemi}{$\checkmark$}

\renewcommand{\CancelColor}{\color{red}}

\newcolumntype{L}[1]{>{\raggedright\let\newline\\\arraybackslash}m{#1}}

\newcolumntype{C}[1]{>{\centering\let\newline\\\arraybackslash}m{#1}}

\newcolumntype{R}[1]{>{\raggedleft\let\newline\\\arraybackslash}m{#1}}

\begin{center}
	\textbf{\LARGE{Parallelization of 3D printer algorithm using Open-CL framework }}\\
	\vspace{7mm}
	\textbf{\large{Sharankumar Narayan Huggi (vipersnh)}}\\
	\vspace{4mm}
	\textbf{\large{University of Michigan}}\\
	\textbf{\large{EECS 587: Parallel Computing}}\\
	\textbf{\large{Profesor: Quentin F. Stout}}\\
	\today
\end{center}

\vspace{7mm}

\section*{\centering Abstract}
3D printers are a pretty common name now-a-days among many engineers and do-it-yourself (DIY) enthusiasts. They come in variety of packages and formats, and print from different plastic materials. There are printers which can even print metal 3D objects. A major problem among the commercially available 3D printers, as well as DIY printers is that they take very long time to print 3D objects. This time not only depends on how much plastic material is infused into the 3D object but also on the complexity of the 3D object. Professor Chinedum Okwudire from the Mechanical Engineering Department at University of Michigan has designed a complex algorithm to reduce the time taken by 3D printers for printing 3D objects up-to 50 percent. However, running the algorithm to generate the sequence for 3D printer itself takes quite a lot of time. In this paper I address the implementation of the complex algorithm using parallel computing and Open-CL frameworks to reduce the algorithm execution time.

\section{Introduction}
Today we live in a world where technology drives our day-to-day activities and our surroundings. 3D printers are a set of technologies which are making big impact on our everyday lives. 3D printers help in rapid prototyping of 3D objects which help designers of products have a first hand experience of their products even before their actual production. They help students take their Computer Aided Design (CAD) models to reality without much effort which was not possible before the invention of 3D printers.

Now that 3D printers have come into existence, human quest for making them faster, more accurate and durable takes them further than what they were designed to be. This process starts by identifying the performance parameters which can be optimized or made faster. One such parameter is time taken for printing a 3D object using a DIY 3D printer. 

A 3D printer mainly consists of motors which drive a print head in 3-dimensional space. The print head is composed of heating chamber which melts plastic material to be drawn in lines to produce a 3D print layer by layer. Another special motor draws the plastic material (which exists in the form of a thread) into the heating chamber to melt newer plastic and flush out molten plastic. The image <TODO> shows a representational 3D printer which can be used to correlate the concepts presented in the paper.

Most of the time consumed in a 3D printer comes from the speed at which the motors run & the speed at which the heating chamber melts plastic. The motors have a physical limit on their speed since they must be controlled at each step. Going beyond their speed limits in software results in them missing software step commands which results in deformed prints. Thus the available 3D printers are shipped with the software under utilizing the speed of motors to be safe from deformed prints. Prof. Okwudire algorithm combines advanced processing into 3D printer software which promises to increase the top speed at which 3D printers operate by pre-calculating jerks produced at top speed and optimizing that part in 3D printer software.

\section{Motivation}
During the research phase of the professor, the algorithm for optimizing the speed of 3D printing was designed initially in MatLab & Simulink software provided by MathWorks Inc. Because MatLab & Simulink are very high level languages they are pretty CPU intensive for the work. Since it is a very inefficient task to run an algorithm in MatLab during production deployment once prototyping in MatLab works as expected, I took up this project to convert the implementation from MatLab to C++ and apply parallel computing algorithms for the processing part to improve the performance. Since I worked in the 3D project team's embedded systems implementation part, I'm very well aware of the concepts used in the whole project. This also forms a great optimization project to understand and implement parallelization concepts since computation time can be optimized using GPU processing.

\section{Related Work}


\section{Architecture}

\section{Implementation}

\section{Validation and Verification}

\section{Parallelization of Algorithm}

\section{Performance Results}

\section{Conclusion}

\section{Future Work}

\section{References\label{sec:references}}

\printbibliography[heading=none]

\end{document}
